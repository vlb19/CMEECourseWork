\documentclass[11pt]{article}
\title {MiniProject: Fitting mechanistic and phenomenological models to functional response curves}
\author{Victoria Blanchard vlb19@ic.ac.uk}
\date{26 February}
\usepackage[margin=2cm]{geometry}
\usepackage{graphicx}
\usepackage{pdfpages}
\usepackage{setspace}
\usepackage[]{lineno}
\usepackage{booktabs}
\usepackage[backend=biber,style=authoryear,bibencoding=utf8]{biblatex}
\setlength{\parindent}{0em}

% Word Count %
\newcommand\wordcount{\documentclass[11pt]{article}
\title {MiniProject: Fitting mechanistic and phenomenological models to functional response curves}
\author{Victoria Blanchard vlb19@ic.ac.uk}
\date{26 February}
\usepackage[margin=2cm]{geometry}
\usepackage{graphicx}
\usepackage{pdfpages}
\usepackage{setspace}
\usepackage[]{lineno}
\usepackage{booktabs}
\usepackage[backend=biber,style=authoryear,bibencoding=utf8]{biblatex}
\setlength{\parindent}{0em}

% Word Count %
\newcommand\wordcount{\documentclass[11pt]{article}
\title {MiniProject: Fitting mechanistic and phenomenological models to functional response curves}
\author{Victoria Blanchard vlb19@ic.ac.uk}
\date{26 February}
\usepackage[margin=2cm]{geometry}
\usepackage{graphicx}
\usepackage{pdfpages}
\usepackage{setspace}
\usepackage[]{lineno}
\usepackage{booktabs}
\usepackage[backend=biber,style=authoryear,bibencoding=utf8]{biblatex}
\setlength{\parindent}{0em}

% Word Count %
\newcommand\wordcount{\documentclass[11pt]{article}
\title {MiniProject: Fitting mechanistic and phenomenological models to functional response curves}
\author{Victoria Blanchard vlb19@ic.ac.uk}
\date{26 February}
\usepackage[margin=2cm]{geometry}
\usepackage{graphicx}
\usepackage{pdfpages}
\usepackage{setspace}
\usepackage[]{lineno}
\usepackage{booktabs}
\usepackage[backend=biber,style=authoryear,bibencoding=utf8]{biblatex}
\setlength{\parindent}{0em}

% Word Count %
\newcommand\wordcount{\input{MiniProjectReport.sum}}


% bibliography secion %

\addbibresource{MResReferences.bib}
\newcommand{\ra}[1]{\renewcommand{\arraystretch}{#1}}

\begin{document}
	
	\begin{titlepage}
		
		
		\centering % this centers everything on the page
		
		%\vspace  % Whitespace at the top of the page 
		
		
		% --------------------------
		%% TITLE
		
		\vspace*{5\baselineskip}
		
		\rule{\textwidth}{1.6pt}\vspace*{-\baselineskip}\vspace*{2pt} % Thick horizontal rule
		\rule{\textwidth}{0.4pt} % Thin horizontal rule
		
		\vspace{0.75\baselineskip} % Whitespace above the title
		
		{\LARGE Mini Project: \\ Fitting mechanistic and phenomenological \\ models to functional response data} 
		
		\vspace{0.75\baselineskip} % Whitespace below the title
		
		\rule{\textwidth}{0.4pt}\vspace*{-\baselineskip}\vspace{3.2pt} 
		\rule{\textwidth}{1.6pt} 
		
		\vspace{2\baselineskip} 
		
		% ---------------------------------
		%% SUPERVISORS & CONTACT EMAIL
		
		Author: \\
		Victoria Blanchard \\
		Imperial College London
		
		\vspace{1.5 \baselineskip} % Whitespace between text
		
		Contact: \\
		vlb19@imperial.ac.uk
		\mbox{}
		\vfill
		\wordcount words
		
	\end{titlepage}
	
	\linenumbers
	\doublespacing
	
	\section*{Introduction}
	
	An understanding of feeding interactions is essential to the field of ecology. Variation in functional traits such as feeding strategy, feeding preference, and mate choice can generate variation in the rate of increase and persistence of populations. The feeding rate of a consumer describes the transfer of biomass between trophic levels and can describe coupled predator-prey abundances in the simplest models (e.g. Lotka 1925). Feeding rate can influence the length of food chains and the distribution of predators through space. Functional responses attempt to describe the relationship between consumption rate and the abundance of the target resource. They arise from biological and physical restraints on consumer-resource interactions and determine the rate of biomass flow between species in ecosystems of all sizes. There have been many phenomenological models for functional responses e.g. (Holling 1965, Lundberg 1988, Ivlev 1961,  Michaelis-Menton 1970). These have revealed patterns within the data that have led to the development of mechanistic models (e.g. Boltzmn-Arrhenius model, Spalinger-Hobbs model 1991) which attempt to validate possible explanations for these patterns. \\
	
	
	\section*{Methods}
	
	
	
	\subsection*{Computing tools}
	
	Python was used for data exploration and analysis. In the initial data exploration, pandas were used for all data frame manipulations and numpy was used for generating nan values. The NLLS fitting script was written in R and used tidyr for data nesting, dplyr used for binding rows between temporary and finalised data frames, and minpack.lm used for all model fitting. Data analysis was completed in python using pandas for reading in csv files and generating crosstab tables for data visualisation. The actual data analysis and subsequent post hoc tests were performed using the pingouin package, and figures were generated using matplotlib.pyplot and seaborn. Finally, the scripts were all compiled using an overarching bash script. 
	
	
	\section*{Results}
	
	\newpage
	\begin{figure}[h!]
		
		\includepdf{../results/HabitatCompare.pdf}
		\caption{	Gantt Chart}
		\label{MRes Gantt}
		
	\end{figure} 
	
	\newpage
	
	\begin{figure}[h!]
	
	\includepdf{../results/PhenOrMec.pdf}
	\caption{	Gantt Chart}
	\label{MRes Gantt}
	
	\end{figure} 

	\newpage
	\begin{thebibliography}{1}
	
	\end{thebibliography}
	
\end{document}
}


% bibliography secion %

\addbibresource{MResReferences.bib}
\newcommand{\ra}[1]{\renewcommand{\arraystretch}{#1}}

\begin{document}
	
	\begin{titlepage}
		
		
		\centering % this centers everything on the page
		
		%\vspace  % Whitespace at the top of the page 
		
		
		% --------------------------
		%% TITLE
		
		\vspace*{5\baselineskip}
		
		\rule{\textwidth}{1.6pt}\vspace*{-\baselineskip}\vspace*{2pt} % Thick horizontal rule
		\rule{\textwidth}{0.4pt} % Thin horizontal rule
		
		\vspace{0.75\baselineskip} % Whitespace above the title
		
		{\LARGE Mini Project: \\ Fitting mechanistic and phenomenological \\ models to functional response data} 
		
		\vspace{0.75\baselineskip} % Whitespace below the title
		
		\rule{\textwidth}{0.4pt}\vspace*{-\baselineskip}\vspace{3.2pt} 
		\rule{\textwidth}{1.6pt} 
		
		\vspace{2\baselineskip} 
		
		% ---------------------------------
		%% SUPERVISORS & CONTACT EMAIL
		
		Author: \\
		Victoria Blanchard \\
		Imperial College London
		
		\vspace{1.5 \baselineskip} % Whitespace between text
		
		Contact: \\
		vlb19@imperial.ac.uk
		\mbox{}
		\vfill
		\wordcount words
		
	\end{titlepage}
	
	\linenumbers
	\doublespacing
	
	\section*{Introduction}
	
	An understanding of feeding interactions is essential to the field of ecology. Variation in functional traits such as feeding strategy, feeding preference, and mate choice can generate variation in the rate of increase and persistence of populations. The feeding rate of a consumer describes the transfer of biomass between trophic levels and can describe coupled predator-prey abundances in the simplest models (e.g. Lotka 1925). Feeding rate can influence the length of food chains and the distribution of predators through space. Functional responses attempt to describe the relationship between consumption rate and the abundance of the target resource. They arise from biological and physical restraints on consumer-resource interactions and determine the rate of biomass flow between species in ecosystems of all sizes. There have been many phenomenological models for functional responses e.g. (Holling 1965, Lundberg 1988, Ivlev 1961,  Michaelis-Menton 1970). These have revealed patterns within the data that have led to the development of mechanistic models (e.g. Boltzmn-Arrhenius model, Spalinger-Hobbs model 1991) which attempt to validate possible explanations for these patterns. \\
	
	
	\section*{Methods}
	
	
	
	\subsection*{Computing tools}
	
	Python was used for data exploration and analysis. In the initial data exploration, pandas were used for all data frame manipulations and numpy was used for generating nan values. The NLLS fitting script was written in R and used tidyr for data nesting, dplyr used for binding rows between temporary and finalised data frames, and minpack.lm used for all model fitting. Data analysis was completed in python using pandas for reading in csv files and generating crosstab tables for data visualisation. The actual data analysis and subsequent post hoc tests were performed using the pingouin package, and figures were generated using matplotlib.pyplot and seaborn. Finally, the scripts were all compiled using an overarching bash script. 
	
	
	\section*{Results}
	
	\newpage
	\begin{figure}[h!]
		
		\includepdf{../results/HabitatCompare.pdf}
		\caption{	Gantt Chart}
		\label{MRes Gantt}
		
	\end{figure} 
	
	\newpage
	
	\begin{figure}[h!]
	
	\includepdf{../results/PhenOrMec.pdf}
	\caption{	Gantt Chart}
	\label{MRes Gantt}
	
	\end{figure} 

	\newpage
	\begin{thebibliography}{1}
	
	\end{thebibliography}
	
\end{document}
}


% bibliography secion %

\addbibresource{MResReferences.bib}
\newcommand{\ra}[1]{\renewcommand{\arraystretch}{#1}}

\begin{document}
	
	\begin{titlepage}
		
		
		\centering % this centers everything on the page
		
		%\vspace  % Whitespace at the top of the page 
		
		
		% --------------------------
		%% TITLE
		
		\vspace*{5\baselineskip}
		
		\rule{\textwidth}{1.6pt}\vspace*{-\baselineskip}\vspace*{2pt} % Thick horizontal rule
		\rule{\textwidth}{0.4pt} % Thin horizontal rule
		
		\vspace{0.75\baselineskip} % Whitespace above the title
		
		{\LARGE Mini Project: \\ Fitting mechanistic and phenomenological \\ models to functional response data} 
		
		\vspace{0.75\baselineskip} % Whitespace below the title
		
		\rule{\textwidth}{0.4pt}\vspace*{-\baselineskip}\vspace{3.2pt} 
		\rule{\textwidth}{1.6pt} 
		
		\vspace{2\baselineskip} 
		
		% ---------------------------------
		%% SUPERVISORS & CONTACT EMAIL
		
		Author: \\
		Victoria Blanchard \\
		Imperial College London
		
		\vspace{1.5 \baselineskip} % Whitespace between text
		
		Contact: \\
		vlb19@imperial.ac.uk
		\mbox{}
		\vfill
		\wordcount words
		
	\end{titlepage}
	
	\linenumbers
	\doublespacing
	
	\section*{Introduction}
	
	An understanding of feeding interactions is essential to the field of ecology. Variation in functional traits such as feeding strategy, feeding preference, and mate choice can generate variation in the rate of increase and persistence of populations. The feeding rate of a consumer describes the transfer of biomass between trophic levels and can describe coupled predator-prey abundances in the simplest models (e.g. Lotka 1925). Feeding rate can influence the length of food chains and the distribution of predators through space. Functional responses attempt to describe the relationship between consumption rate and the abundance of the target resource. They arise from biological and physical restraints on consumer-resource interactions and determine the rate of biomass flow between species in ecosystems of all sizes. There have been many phenomenological models for functional responses e.g. (Holling 1965, Lundberg 1988, Ivlev 1961,  Michaelis-Menton 1970). These have revealed patterns within the data that have led to the development of mechanistic models (e.g. Boltzmn-Arrhenius model, Spalinger-Hobbs model 1991) which attempt to validate possible explanations for these patterns. \\
	
	
	\section*{Methods}
	
	
	
	\subsection*{Computing tools}
	
	Python was used for data exploration and analysis. In the initial data exploration, pandas were used for all data frame manipulations and numpy was used for generating nan values. The NLLS fitting script was written in R and used tidyr for data nesting, dplyr used for binding rows between temporary and finalised data frames, and minpack.lm used for all model fitting. Data analysis was completed in python using pandas for reading in csv files and generating crosstab tables for data visualisation. The actual data analysis and subsequent post hoc tests were performed using the pingouin package, and figures were generated using matplotlib.pyplot and seaborn. Finally, the scripts were all compiled using an overarching bash script. 
	
	
	\section*{Results}
	
	\newpage
	\begin{figure}[h!]
		
		\includepdf{../results/HabitatCompare.pdf}
		\caption{	Gantt Chart}
		\label{MRes Gantt}
		
	\end{figure} 
	
	\newpage
	
	\begin{figure}[h!]
	
	\includepdf{../results/PhenOrMec.pdf}
	\caption{	Gantt Chart}
	\label{MRes Gantt}
	
	\end{figure} 

	\newpage
	\begin{thebibliography}{1}
	
	\end{thebibliography}
	
\end{document}
}


% bibliography secion %

\addbibresource{MResReferences.bib}
\newcommand{\ra}[1]{\renewcommand{\arraystretch}{#1}}

\begin{document}
	
	\begin{titlepage}
		
		
		\centering % this centers everything on the page
		
		%\vspace  % Whitespace at the top of the page 
		
		
		% --------------------------
		%% TITLE
		
		\vspace*{5\baselineskip}
		
		\rule{\textwidth}{1.6pt}\vspace*{-\baselineskip}\vspace*{2pt} % Thick horizontal rule
		\rule{\textwidth}{0.4pt} % Thin horizontal rule
		
		\vspace{0.75\baselineskip} % Whitespace above the title
		
		{\LARGE Mini Project: \\ Fitting mechanistic and phenomenological \\ models to functional response data} 
		
		\vspace{0.75\baselineskip} % Whitespace below the title
		
		\rule{\textwidth}{0.4pt}\vspace*{-\baselineskip}\vspace{3.2pt} 
		\rule{\textwidth}{1.6pt} 
		
		\vspace{2\baselineskip} 
		
		% ---------------------------------
		%% SUPERVISORS & CONTACT EMAIL
		
		Author: \\
		Victoria Blanchard \\
		Imperial College London
		
		\vspace{1.5 \baselineskip} % Whitespace between text
		
		Contact: \\
		vlb19@imperial.ac.uk
		\mbox{}
		\vfill
		\wordcount words
		
	\end{titlepage}
	
	\linenumbers
	\doublespacing
	
	\section*{Introduction}
	
	An understanding of feeding interactions is essential to the field of ecology. Variation in functional traits such as feeding strategy, feeding preference, and mate choice can generate variation in the rate of increase and persistence of populations. The feeding rate of a consumer describes the transfer of biomass between trophic levels and can describe coupled predator-prey abundances in the simplest models (e.g. Lotka 1925). Feeding rate can influence the length of food chains and the distribution of predators through space. Functional responses attempt to describe the relationship between consumption rate and the abundance of the target resource. They arise from biological and physical restraints on consumer-resource interactions and determine the rate of biomass flow between species in ecosystems of all sizes. There have been many phenomenological models for functional responses e.g. (Holling 1965, Lundberg 1988, Ivlev 1961,  Michaelis-Menton 1970). These have revealed patterns within the data that have led to the development of mechanistic models (e.g. Boltzmn-Arrhenius model, Spalinger-Hobbs model 1991) which attempt to validate possible explanations for these patterns. \\
	
	
	\section*{Methods}
	
	
	
	\subsection*{Computing tools}
	
	Python was used for data exploration and analysis. In the initial data exploration, pandas were used for all data frame manipulations and numpy was used for generating nan values. The NLLS fitting script was written in R and used tidyr for data nesting, dplyr used for binding rows between temporary and finalised data frames, and minpack.lm used for all model fitting. Data analysis was completed in python using pandas for reading in csv files and generating crosstab tables for data visualisation. The actual data analysis and subsequent post hoc tests were performed using the pingouin package, and figures were generated using matplotlib.pyplot and seaborn. Finally, the scripts were all compiled using an overarching bash script. 
	
	
	\section*{Results}
	
	\newpage
	\begin{figure}[h!]
		
		\includepdf{../results/HabitatCompare.pdf}
		\caption{	Gantt Chart}
		\label{MRes Gantt}
		
	\end{figure} 
	
	\newpage
	
	\begin{figure}[h!]
	
	\includepdf{../results/PhenOrMec.pdf}
	\caption{	Gantt Chart}
	\label{MRes Gantt}
	
	\end{figure} 

	\newpage
	\begin{thebibliography}{1}
	
	\end{thebibliography}
	
\end{document}
